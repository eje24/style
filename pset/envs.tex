\documentclass{scrartcl}
\usepackage{pset}

\title{Basic Documentation}


\begin{document}

\maketitle

This file contains basic documentation for most environments present in \verb#pset.sty#. For stuff on package options, headings, and more, feel free to ask me, or see \href{https://github.com/eje24/style}{here}. Feel free to fork into your own personal style file and make additional modifications as you see fit.

\tableofcontents


\section{Standard Environments}
\begin{theorem}
    This is a theorem!  To use me, type \verb#\begin{theorem} ... \end{theorem}#.
\end{theorem}
\begin{lemma}
    This is a lemma!  To use me, type \verb#\begin{lemma} ... \end{lemma}#.
\end{lemma}
\begin{proposition}
    This is a proposition! To use me, type \verb#\begin{proposition} ... \end{proposition}#.
\end{proposition}
\begin{corollary}
    This is a corollary! To use me, type \verb#\begin{corollary} ... \end{corollary}#.
\end{corollary}
\begin{conjecture}
    This is a conjecture! To use me, type \verb#\begin{conjecture} ... \end{conjecture}#.
\end{conjecture}
\begin{problem}
    This is a problem! To use me, type \verb#\begin{problem} ... \end{problem}#.
\end{problem}
\begin{example}
    This is a example! To use me, type \verb#\begin{example} ... \end{example}#.
\end{example}
\begin{question}
    This is a question! To use me, type \verb#\begin{question} ... \end{question}#.
\end{question}
\begin{remark}
    This is a remark! To use me, type \verb#\begin{remark} ... \end{remark}#.
\end{remark}

\newpage

\section{Boxed Environments}
% Boxed environments are partitioned into three categories. Boxed environments within a given category are all the same color. You can set this color to anything from the \href{https://en.wikibooks.org/wiki/LaTeX/Colors}{xcolor package} here. See below for default colors and directions for customizing colors.

\subsection{Plain Category}
Default color is \textbf{CornflowerBlue}. Color can be set with \verb#\setplaincolor{COLOR_NAME}#.

\begin{theorembox}
    This is a theorembox! To use me, type \verb#\begin{theorembox} ... \end{theorembox}#.
\end{theorembox}
\begin{lemmabox}
    This is a lemmabox! To use me, type \verb#\begin{lemmabox} ... \end{lemmabox}#.
\end{lemmabox}
\begin{propbox}
    This is a propbox! To use me, type \verb#\begin{propbox} ... \end{propbox}#.
\end{propbox}

\subsection{Def Category}

Default color is \textbf{Emerald}. Color can be set with \verb#\setdefcolor{COLOR_NAME}#.

\begin{corollarybox}
    This is a corollarybox! To use me, type \verb#\begin{corollarybox} ... \end{corollarybox}#.
\end{corollarybox}
\begin{conjecturebox}
    This is a conjecturebox! To use me, type \verb#\begin{conjecturebox} ... \end{conjecturebox}#.
\end{conjecturebox}
\begin{claimbox}
    This is a claimbox! To use me, type \verb#\begin{claimbox} ... \end{claimbox}#.
\end{claimbox}
\begin{problembox}
    This is a problembox!  To use me, type \verb#\begin{problembox} ... \end{problembox}#.
\end{problembox}
\begin{examplebox}
    This is a examplebox!  To use me, type \verb#\begin{examplebox} ... \end{examplebox}#.
\end{examplebox}
\begin{questionbox}
    This is a question!  To use me, type \verb#\begin{questionbox} ... \end{questionbox}#.
\end{questionbox}

\subsection{Remark Category}

Default color is \textbf{Periwinkle}. Color can be set with \verb#\setremarkcolor{COLOR_NAME}#. 

\begin{remarkbox}
    This is a remarkbox!  To use me, type \verb#\begin{remarkbox} ... \end{remarkbox}#.
\end{remarkbox}

\subsection{Other}

\begin{miscbox}
    This is miscbox! To use me, type \verb#\begin{miscbox} ... \end{miscbox}#. To specify a color, type \verb#\begin{miscbox}[COLOR_NAME] ... \end{miscbox}#. The default is Fuchsia, and this will be used if no other color is specified. To change the default, you can use \verb#\setmisccolor{COLOR_NAME}#.
\end{miscbox}


\newpage

\section{Proof Environments}
\begin{proof}
    Test proof! To use me, type \verb#\begin{proof} ... \end{proof}#.
\end{proof}
\begin{subproof}
    Test subproof! To use me, type \verb#\begin{subproof} ... \end{subproof}#.
\end{subproof}
\begin{proof*}{My Proof}
    Test proof*! To use me, type \verb#\begin{proof*}{PROOF_NAME} ... \end{proof*}#.
\end{proof*}
\begin{subproof*}{My Subproof}
    Test subproof*! To use me, type \verb#\begin{subproof*}{PROOF_NAME} ... \end{subproof*}#.
\end{subproof*}



\end{document}