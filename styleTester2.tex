\documentclass{scrartcl}
\usepackage[pset]{eje}


\setname{Ben Bitdiddle}
\setpsetclass{18.701}

\setpsetstyle{Series Test}


\title{Series Test}
\begin{document}



\maketitle

\begin{abstract}
The text on this page is taken from \href{http://www-math.mit.edu/~poonen/715/real_representations.pdf}{this document}, for font testing purposes. The goal of these notes is to explain the classification of real representations of a finite
group. Throughout, $G$ is a finite group, $W$ is a \RR-vector space or \RR G-module, and $V$ is a
\CC-vector space or \CC G-module (except in Section 2, where $V$ is over any field). Vector spaces
and representations are assumed to be finite-dimensional.

\end{abstract}

\newpset

\newpsetproblem{Constructions}

The goal of these notes is to explain the classification of real representations of a finite
group. Throughout, $G$ is a finite group, $W$ is a \RR-vector space or \RR G-module, and $V$ is a
\CC-vector space or \CC G-module (except in Section 2, where $V$ is over any field). Vector spaces
and representations are assumed to be finite-dimensional.

To get from $\RR^n$ to $\CC^n$, we can tensor with \CC. In a more coordinate-free
manner, if $W$ is an \RR-vector space, then its \obold{complexification} $W_\CC := W \otimes_\RR \CC$ is a \CC-vector
space. We can view $W$ as an \RR-subspace of $W_\CC$ by identifying each $w \in W$ with $w \otimes 1 \in W_\CC$.
Then an \RR-basis of $W$ is also a \CC-basis of $W_\CC$. In particular, $W_\CC$ has the same dimension as
$W$ (but is a vector space over a different field).

Conversely, we can view \CCn as $\RR^{2n}$ if we forget how to multiply by complex scalars that are not real. In a more coordinate-free manner, then its \obold{restriction of scalars} is the \RR-vector space $_\RR V$ with the same underlying abelian group but with only scalar multiplication by real numbers. If $v_1,\dots,v_n$ is a \CC-basis of $V$, then $v_1, iv_1,\dots,v_n,iv_n$ is an \RR-basis of $_\RR V$. In particular, $\dim(_\RR V) = 2\dim V$.

Also, if $V$ is a \CC-vector space, then the \obold{complex conjugate vector space} $\ol{V}$ has the same underlying group but a new scalar multiplication $\cdot$ defined by $\lambda\cdot v := \ol{\lambda}v$, where $\ol{\lambda}v$ is defined using the origial scalar multiplication.

Complexification and restriction of scalars are not inverse constructions. Instead:
\begin{proposition}[Complexification and restriction of scalars.]
    \hfill
    \begin{enumerate}[font=\normalfont]
        \item If $V$ is a \CC-vector space, then the map \begin{align*}
            (_\RR V)_\CC &\rightarrow V \oplus \ol{V} \\
            v\otimes c &\mapsto (cv,\ol{c}v) \\
        \end{align*} is an isomorphism of \CC-vector spaces.
        \item If $W$ is an \RR-vector space, then \[_\RR(W_\CC) \simeq W\oplus W.\]
    \end{enumerate}
\end{proposition}

\newpsetproblem{Constructions}

The goal of these notes is to explain the classification of real representations of a finite
group. Throughout, $G$ is a finite group, $W$ is a \RR-vector space or \RR G-module, and $V$ is a
\CC-vector space or \CC G-module (except in Section 2, where $V$ is over any field). Vector spaces
and representations are assumed to be finite-dimensional.

To get from $\RR^n$ to $\CC^n$, we can tensor with \CC. In a more coordinate-free
manner, if $W$ is an \RR-vector space, then its \obold{complexification} $W_\CC := W \otimes_\RR \CC$ is a \CC-vector
space. We can view $W$ as an \RR-subspace of $W_\CC$ by identifying each $w \in W$ with $w \otimes 1 \in W_\CC$.
Then an \RR-basis of $W$ is also a \CC-basis of $W_\CC$. In particular, $W_\CC$ has the same dimension as
$W$ (but is a vector space over a different field).

Conversely, we can view \CCn as $\RR^{2n}$ if we forget how to multiply by complex scalars that are not real. In a more coordinate-free manner, then its \obold{restriction of scalars} is the \RR-vector space $_\RR V$ with the same underlying abelian group but with only scalar multiplication by real numbers. If $v_1,\dots,v_n$ is a \CC-basis of $V$, then $v_1, iv_1,\dots,v_n,iv_n$ is an \RR-basis of $_\RR V$. In particular, $\dim(_\RR V) = 2\dim V$.

Also, if $V$ is a \CC-vector space, then the \obold{complex conjugate vector space} $\ol{V}$ has the same underlying group but a new scalar multiplication $\cdot$ defined by $\lambda\cdot v := \ol{\lambda}v$, where $\ol{\lambda}v$ is defined using the origial scalar multiplication.

Complexification and restriction of scalars are not inverse constructions. Instead:
\begin{proposition}[Complexification and restriction of scalars.]
    \hfill
    \begin{enumerate}[font=\normalfont]
        \item If $V$ is a \CC-vector space, then the map \begin{align*}
            (_\RR V)_\CC &\rightarrow V \oplus \ol{V} \\
            v\otimes c &\mapsto (cv,\ol{c}v) \\
        \end{align*} is an isomorphism of \CC-vector spaces.
        \item If $W$ is an \RR-vector space, then \[_\RR(W_\CC) \simeq W\oplus W.\]
    \end{enumerate}
\end{proposition}

\newpsetproblem{Constructions}

The goal of these notes is to explain the classification of real representations of a finite
group. Throughout, $G$ is a finite group, $W$ is a \RR-vector space or \RR G-module, and $V$ is a
\CC-vector space or \CC G-module (except in Section 2, where $V$ is over any field). Vector spaces
and representations are assumed to be finite-dimensional.

To get from $\RR^n$ to $\CC^n$, we can tensor with \CC. In a more coordinate-free
manner, if $W$ is an \RR-vector space, then its \obold{complexification} $W_\CC := W \otimes_\RR \CC$ is a \CC-vector
space. We can view $W$ as an \RR-subspace of $W_\CC$ by identifying each $w \in W$ with $w \otimes 1 \in W_\CC$.
Then an \RR-basis of $W$ is also a \CC-basis of $W_\CC$. In particular, $W_\CC$ has the same dimension as
$W$ (but is a vector space over a different field).

Conversely, we can view \CCn as $\RR^{2n}$ if we forget how to multiply by complex scalars that are not real. In a more coordinate-free manner, then its \obold{restriction of scalars} is the \RR-vector space $_\RR V$ with the same underlying abelian group but with only scalar multiplication by real numbers. If $v_1,\dots,v_n$ is a \CC-basis of $V$, then $v_1, iv_1,\dots,v_n,iv_n$ is an \RR-basis of $_\RR V$. In particular, $\dim(_\RR V) = 2\dim V$.

Also, if $V$ is a \CC-vector space, then the \obold{complex conjugate vector space} $\ol{V}$ has the same underlying group but a new scalar multiplication $\cdot$ defined by $\lambda\cdot v := \ol{\lambda}v$, where $\ol{\lambda}v$ is defined using the origial scalar multiplication.

Complexification and restriction of scalars are not inverse constructions. Instead:
\begin{proposition}[Complexification and restriction of scalars.]
    \hfill
    \begin{enumerate}[font=\normalfont]
        \item If $V$ is a \CC-vector space, then the map \begin{align*}
            (_\RR V)_\CC &\rightarrow V \oplus \ol{V} \\
            v\otimes c &\mapsto (cv,\ol{c}v) \\
        \end{align*} is an isomorphism of \CC-vector spaces.
        \item If $W$ is an \RR-vector space, then \[_\RR(W_\CC) \simeq W\oplus W.\]
    \end{enumerate}
\end{proposition}

\newpsetproblem{Constructions}

The goal of these notes is to explain the classification of real representations of a finite
group. Throughout, $G$ is a finite group, $W$ is a \RR-vector space or \RR G-module, and $V$ is a
\CC-vector space or \CC G-module (except in Section 2, where $V$ is over any field). Vector spaces
and representations are assumed to be finite-dimensional.

To get from $\RR^n$ to $\CC^n$, we can tensor with \CC. In a more coordinate-free
manner, if $W$ is an \RR-vector space, then its \obold{complexification} $W_\CC := W \otimes_\RR \CC$ is a \CC-vector
space. We can view $W$ as an \RR-subspace of $W_\CC$ by identifying each $w \in W$ with $w \otimes 1 \in W_\CC$.
Then an \RR-basis of $W$ is also a \CC-basis of $W_\CC$. In particular, $W_\CC$ has the same dimension as
$W$ (but is a vector space over a different field).

Conversely, we can view \CCn as $\RR^{2n}$ if we forget how to multiply by complex scalars that are not real. In a more coordinate-free manner, then its \obold{restriction of scalars} is the \RR-vector space $_\RR V$ with the same underlying abelian group but with only scalar multiplication by real numbers. If $v_1,\dots,v_n$ is a \CC-basis of $V$, then $v_1, iv_1,\dots,v_n,iv_n$ is an \RR-basis of $_\RR V$. In particular, $\dim(_\RR V) = 2\dim V$.

Also, if $V$ is a \CC-vector space, then the \obold{complex conjugate vector space} $\ol{V}$ has the same underlying group but a new scalar multiplication $\cdot$ defined by $\lambda\cdot v := \ol{\lambda}v$, where $\ol{\lambda}v$ is defined using the origial scalar multiplication.

Complexification and restriction of scalars are not inverse constructions. Instead:
\begin{proposition}[Complexification and restriction of scalars.]
    \hfill
    \begin{enumerate}[font=\normalfont]
        \item If $V$ is a \CC-vector space, then the map \begin{align*}
            (_\RR V)_\CC &\rightarrow V \oplus \ol{V} \\
            v\otimes c &\mapsto (cv,\ol{c}v) \\
        \end{align*} is an isomorphism of \CC-vector spaces.
        \item If $W$ is an \RR-vector space, then \[_\RR(W_\CC) \simeq W\oplus W.\]
    \end{enumerate}
\end{proposition}

\newpset

\newpsetproblem{Constructions}

The goal of these notes is to explain the classification of real representations of a finite
group. Throughout, $G$ is a finite group, $W$ is a \RR-vector space or \RR G-module, and $V$ is a
\CC-vector space or \CC G-module (except in Section 2, where $V$ is over any field). Vector spaces
and representations are assumed to be finite-dimensional.

To get from $\RR^n$ to $\CC^n$, we can tensor with \CC. In a more coordinate-free
manner, if $W$ is an \RR-vector space, then its \obold{complexification} $W_\CC := W \otimes_\RR \CC$ is a \CC-vector
space. We can view $W$ as an \RR-subspace of $W_\CC$ by identifying each $w \in W$ with $w \otimes 1 \in W_\CC$.
Then an \RR-basis of $W$ is also a \CC-basis of $W_\CC$. In particular, $W_\CC$ has the same dimension as
$W$ (but is a vector space over a different field).

Conversely, we can view \CCn as $\RR^{2n}$ if we forget how to multiply by complex scalars that are not real. In a more coordinate-free manner, then its \obold{restriction of scalars} is the \RR-vector space $_\RR V$ with the same underlying abelian group but with only scalar multiplication by real numbers. If $v_1,\dots,v_n$ is a \CC-basis of $V$, then $v_1, iv_1,\dots,v_n,iv_n$ is an \RR-basis of $_\RR V$. In particular, $\dim(_\RR V) = 2\dim V$.

Also, if $V$ is a \CC-vector space, then the \obold{complex conjugate vector space} $\ol{V}$ has the same underlying group but a new scalar multiplication $\cdot$ defined by $\lambda\cdot v := \ol{\lambda}v$, where $\ol{\lambda}v$ is defined using the origial scalar multiplication.

Complexification and restriction of scalars are not inverse constructions. Instead:
\begin{proposition}[Complexification and restriction of scalars.]
    \hfill
    \begin{enumerate}[font=\normalfont]
        \item If $V$ is a \CC-vector space, then the map \begin{align*}
            (_\RR V)_\CC &\rightarrow V \oplus \ol{V} \\
            v\otimes c &\mapsto (cv,\ol{c}v) \\
        \end{align*} is an isomorphism of \CC-vector spaces.
        \item If $W$ is an \RR-vector space, then \[_\RR(W_\CC) \simeq W\oplus W.\]
    \end{enumerate}
\end{proposition}

\newpsetproblem{Constructions}

The goal of these notes is to explain the classification of real representations of a finite
group. Throughout, $G$ is a finite group, $W$ is a \RR-vector space or \RR G-module, and $V$ is a
\CC-vector space or \CC G-module (except in Section 2, where $V$ is over any field). Vector spaces
and representations are assumed to be finite-dimensional.

To get from $\RR^n$ to $\CC^n$, we can tensor with \CC. In a more coordinate-free
manner, if $W$ is an \RR-vector space, then its \obold{complexification} $W_\CC := W \otimes_\RR \CC$ is a \CC-vector
space. We can view $W$ as an \RR-subspace of $W_\CC$ by identifying each $w \in W$ with $w \otimes 1 \in W_\CC$.
Then an \RR-basis of $W$ is also a \CC-basis of $W_\CC$. In particular, $W_\CC$ has the same dimension as
$W$ (but is a vector space over a different field).

Conversely, we can view \CCn as $\RR^{2n}$ if we forget how to multiply by complex scalars that are not real. In a more coordinate-free manner, then its \obold{restriction of scalars} is the \RR-vector space $_\RR V$ with the same underlying abelian group but with only scalar multiplication by real numbers. If $v_1,\dots,v_n$ is a \CC-basis of $V$, then $v_1, iv_1,\dots,v_n,iv_n$ is an \RR-basis of $_\RR V$. In particular, $\dim(_\RR V) = 2\dim V$.

Also, if $V$ is a \CC-vector space, then the \obold{complex conjugate vector space} $\ol{V}$ has the same underlying group but a new scalar multiplication $\cdot$ defined by $\lambda\cdot v := \ol{\lambda}v$, where $\ol{\lambda}v$ is defined using the origial scalar multiplication.

Complexification and restriction of scalars are not inverse constructions. Instead:
\begin{proposition}[Complexification and restriction of scalars.]
    \hfill
    \begin{enumerate}[font=\normalfont]
        \item If $V$ is a \CC-vector space, then the map \begin{align*}
            (_\RR V)_\CC &\rightarrow V \oplus \ol{V} \\
            v\otimes c &\mapsto (cv,\ol{c}v) \\
        \end{align*} is an isomorphism of \CC-vector spaces.
        \item If $W$ is an \RR-vector space, then \[_\RR(W_\CC) \simeq W\oplus W.\]
    \end{enumerate}
\end{proposition}

\newpsetproblem{Constructions}

The goal of these notes is to explain the classification of real representations of a finite
group. Throughout, $G$ is a finite group, $W$ is a \RR-vector space or \RR G-module, and $V$ is a
\CC-vector space or \CC G-module (except in Section 2, where $V$ is over any field). Vector spaces
and representations are assumed to be finite-dimensional.

To get from $\RR^n$ to $\CC^n$, we can tensor with \CC. In a more coordinate-free
manner, if $W$ is an \RR-vector space, then its \obold{complexification} $W_\CC := W \otimes_\RR \CC$ is a \CC-vector
space. We can view $W$ as an \RR-subspace of $W_\CC$ by identifying each $w \in W$ with $w \otimes 1 \in W_\CC$.
Then an \RR-basis of $W$ is also a \CC-basis of $W_\CC$. In particular, $W_\CC$ has the same dimension as
$W$ (but is a vector space over a different field).

Conversely, we can view \CCn as $\RR^{2n}$ if we forget how to multiply by complex scalars that are not real. In a more coordinate-free manner, then its \obold{restriction of scalars} is the \RR-vector space $_\RR V$ with the same underlying abelian group but with only scalar multiplication by real numbers. If $v_1,\dots,v_n$ is a \CC-basis of $V$, then $v_1, iv_1,\dots,v_n,iv_n$ is an \RR-basis of $_\RR V$. In particular, $\dim(_\RR V) = 2\dim V$.

Also, if $V$ is a \CC-vector space, then the \obold{complex conjugate vector space} $\ol{V}$ has the same underlying group but a new scalar multiplication $\cdot$ defined by $\lambda\cdot v := \ol{\lambda}v$, where $\ol{\lambda}v$ is defined using the origial scalar multiplication.

Complexification and restriction of scalars are not inverse constructions. Instead:
\begin{proposition}[Complexification and restriction of scalars.]
    \hfill
    \begin{enumerate}[font=\normalfont]
        \item If $V$ is a \CC-vector space, then the map \begin{align*}
            (_\RR V)_\CC &\rightarrow V \oplus \ol{V} \\
            v\otimes c &\mapsto (cv,\ol{c}v) \\
        \end{align*} is an isomorphism of \CC-vector spaces.
        \item If $W$ is an \RR-vector space, then \[_\RR(W_\CC) \simeq W\oplus W.\]
    \end{enumerate}
\end{proposition}

\newpsetproblem{Constructions}

The goal of these notes is to explain the classification of real representations of a finite
group. Throughout, $G$ is a finite group, $W$ is a \RR-vector space or \RR G-module, and $V$ is a
\CC-vector space or \CC G-module (except in Section 2, where $V$ is over any field). Vector spaces
and representations are assumed to be finite-dimensional.

To get from $\RR^n$ to $\CC^n$, we can tensor with \CC. In a more coordinate-free
manner, if $W$ is an \RR-vector space, then its \obold{complexification} $W_\CC := W \otimes_\RR \CC$ is a \CC-vector
space. We can view $W$ as an \RR-subspace of $W_\CC$ by identifying each $w \in W$ with $w \otimes 1 \in W_\CC$.
Then an \RR-basis of $W$ is also a \CC-basis of $W_\CC$. In particular, $W_\CC$ has the same dimension as
$W$ (but is a vector space over a different field).

Conversely, we can view \CCn as $\RR^{2n}$ if we forget how to multiply by complex scalars that are not real. In a more coordinate-free manner, then its \obold{restriction of scalars} is the \RR-vector space $_\RR V$ with the same underlying abelian group but with only scalar multiplication by real numbers. If $v_1,\dots,v_n$ is a \CC-basis of $V$, then $v_1, iv_1,\dots,v_n,iv_n$ is an \RR-basis of $_\RR V$. In particular, $\dim(_\RR V) = 2\dim V$.

Also, if $V$ is a \CC-vector space, then the \obold{complex conjugate vector space} $\ol{V}$ has the same underlying group but a new scalar multiplication $\cdot$ defined by $\lambda\cdot v := \ol{\lambda}v$, where $\ol{\lambda}v$ is defined using the origial scalar multiplication.

Complexification and restriction of scalars are not inverse constructions. Instead:
\begin{proposition}[Complexification and restriction of scalars.]
    \hfill
    \begin{enumerate}[font=\normalfont]
        \item If $V$ is a \CC-vector space, then the map \begin{align*}
            (_\RR V)_\CC &\rightarrow V \oplus \ol{V} \\
            v\otimes c &\mapsto (cv,\ol{c}v) \\
        \end{align*} is an isomorphism of \CC-vector spaces.
        \item If $W$ is an \RR-vector space, then \[_\RR(W_\CC) \simeq W\oplus W.\]
    \end{enumerate}
\end{proposition}

\newpset

\newpsetproblem{Constructions}

The goal of these notes is to explain the classification of real representations of a finite
group. Throughout, $G$ is a finite group, $W$ is a \RR-vector space or \RR G-module, and $V$ is a
\CC-vector space or \CC G-module (except in Section 2, where $V$ is over any field). Vector spaces
and representations are assumed to be finite-dimensional.

To get from $\RR^n$ to $\CC^n$, we can tensor with \CC. In a more coordinate-free
manner, if $W$ is an \RR-vector space, then its \obold{complexification} $W_\CC := W \otimes_\RR \CC$ is a \CC-vector
space. We can view $W$ as an \RR-subspace of $W_\CC$ by identifying each $w \in W$ with $w \otimes 1 \in W_\CC$.
Then an \RR-basis of $W$ is also a \CC-basis of $W_\CC$. In particular, $W_\CC$ has the same dimension as
$W$ (but is a vector space over a different field).

Conversely, we can view \CCn as $\RR^{2n}$ if we forget how to multiply by complex scalars that are not real. In a more coordinate-free manner, then its \obold{restriction of scalars} is the \RR-vector space $_\RR V$ with the same underlying abelian group but with only scalar multiplication by real numbers. If $v_1,\dots,v_n$ is a \CC-basis of $V$, then $v_1, iv_1,\dots,v_n,iv_n$ is an \RR-basis of $_\RR V$. In particular, $\dim(_\RR V) = 2\dim V$.

Also, if $V$ is a \CC-vector space, then the \obold{complex conjugate vector space} $\ol{V}$ has the same underlying group but a new scalar multiplication $\cdot$ defined by $\lambda\cdot v := \ol{\lambda}v$, where $\ol{\lambda}v$ is defined using the origial scalar multiplication.

Complexification and restriction of scalars are not inverse constructions. Instead:
\begin{proposition}[Complexification and restriction of scalars.]
    \hfill
    \begin{enumerate}[font=\normalfont]
        \item If $V$ is a \CC-vector space, then the map \begin{align*}
            (_\RR V)_\CC &\rightarrow V \oplus \ol{V} \\
            v\otimes c &\mapsto (cv,\ol{c}v) \\
        \end{align*} is an isomorphism of \CC-vector spaces.
        \item If $W$ is an \RR-vector space, then \[_\RR(W_\CC) \simeq W\oplus W.\]
    \end{enumerate}
\end{proposition}

\newpsetproblem{Constructions}

The goal of these notes is to explain the classification of real representations of a finite
group. Throughout, $G$ is a finite group, $W$ is a \RR-vector space or \RR G-module, and $V$ is a
\CC-vector space or \CC G-module (except in Section 2, where $V$ is over any field). Vector spaces
and representations are assumed to be finite-dimensional.

To get from $\RR^n$ to $\CC^n$, we can tensor with \CC. In a more coordinate-free
manner, if $W$ is an \RR-vector space, then its \obold{complexification} $W_\CC := W \otimes_\RR \CC$ is a \CC-vector
space. We can view $W$ as an \RR-subspace of $W_\CC$ by identifying each $w \in W$ with $w \otimes 1 \in W_\CC$.
Then an \RR-basis of $W$ is also a \CC-basis of $W_\CC$. In particular, $W_\CC$ has the same dimension as
$W$ (but is a vector space over a different field).

Conversely, we can view \CCn as $\RR^{2n}$ if we forget how to multiply by complex scalars that are not real. In a more coordinate-free manner, then its \obold{restriction of scalars} is the \RR-vector space $_\RR V$ with the same underlying abelian group but with only scalar multiplication by real numbers. If $v_1,\dots,v_n$ is a \CC-basis of $V$, then $v_1, iv_1,\dots,v_n,iv_n$ is an \RR-basis of $_\RR V$. In particular, $\dim(_\RR V) = 2\dim V$.

Also, if $V$ is a \CC-vector space, then the \obold{complex conjugate vector space} $\ol{V}$ has the same underlying group but a new scalar multiplication $\cdot$ defined by $\lambda\cdot v := \ol{\lambda}v$, where $\ol{\lambda}v$ is defined using the origial scalar multiplication.

Complexification and restriction of scalars are not inverse constructions. Instead:
\begin{proposition}[Complexification and restriction of scalars.]
    \hfill
    \begin{enumerate}[font=\normalfont]
        \item If $V$ is a \CC-vector space, then the map \begin{align*}
            (_\RR V)_\CC &\rightarrow V \oplus \ol{V} \\
            v\otimes c &\mapsto (cv,\ol{c}v) \\
        \end{align*} is an isomorphism of \CC-vector spaces.
        \item If $W$ is an \RR-vector space, then \[_\RR(W_\CC) \simeq W\oplus W.\]
    \end{enumerate}
\end{proposition}

\newpsetproblem{Constructions}

The goal of these notes is to explain the classification of real representations of a finite
group. Throughout, $G$ is a finite group, $W$ is a \RR-vector space or \RR G-module, and $V$ is a
\CC-vector space or \CC G-module (except in Section 2, where $V$ is over any field). Vector spaces
and representations are assumed to be finite-dimensional.

To get from $\RR^n$ to $\CC^n$, we can tensor with \CC. In a more coordinate-free
manner, if $W$ is an \RR-vector space, then its \obold{complexification} $W_\CC := W \otimes_\RR \CC$ is a \CC-vector
space. We can view $W$ as an \RR-subspace of $W_\CC$ by identifying each $w \in W$ with $w \otimes 1 \in W_\CC$.
Then an \RR-basis of $W$ is also a \CC-basis of $W_\CC$. In particular, $W_\CC$ has the same dimension as
$W$ (but is a vector space over a different field).

Conversely, we can view \CCn as $\RR^{2n}$ if we forget how to multiply by complex scalars that are not real. In a more coordinate-free manner, then its \obold{restriction of scalars} is the \RR-vector space $_\RR V$ with the same underlying abelian group but with only scalar multiplication by real numbers. If $v_1,\dots,v_n$ is a \CC-basis of $V$, then $v_1, iv_1,\dots,v_n,iv_n$ is an \RR-basis of $_\RR V$. In particular, $\dim(_\RR V) = 2\dim V$.

Also, if $V$ is a \CC-vector space, then the \obold{complex conjugate vector space} $\ol{V}$ has the same underlying group but a new scalar multiplication $\cdot$ defined by $\lambda\cdot v := \ol{\lambda}v$, where $\ol{\lambda}v$ is defined using the origial scalar multiplication.

Complexification and restriction of scalars are not inverse constructions. Instead:
\begin{proposition}[Complexification and restriction of scalars.]
    \hfill
    \begin{enumerate}[font=\normalfont]
        \item If $V$ is a \CC-vector space, then the map \begin{align*}
            (_\RR V)_\CC &\rightarrow V \oplus \ol{V} \\
            v\otimes c &\mapsto (cv,\ol{c}v) \\
        \end{align*} is an isomorphism of \CC-vector spaces.
        \item If $W$ is an \RR-vector space, then \[_\RR(W_\CC) \simeq W\oplus W.\]
    \end{enumerate}
\end{proposition}


\end{document}



