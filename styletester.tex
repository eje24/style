\documentclass{scrartcl}
\usepackage[nodate]{mrk}



\setlength{\oddsidemargin}{.0in}
\setlength{\evensidemargin}{.0in}
\setlength{\textwidth}{6.5in}
\setlength{\topmargin}{-.3in}
\setlength{\headsep}{.20in}
\setlength{\textheight}{9.in}

%These commands deal with theorem-like environments (i.e., italic)
\theoremstyle{plain}
\newtheorem{theorem}{Theorem}[section]
\newtheorem{corollary}[theorem]{Corollary}
\newtheorem{lemma}[theorem]{Lemma}
\newtheorem{conjecture}[theorem]{Conjecture}

%These deal with definition-like environments (i.e., non-italic)
\theoremstyle{definition}
\newtheorem{definition}[theorem]{Definition}
\newtheorem{example}[theorem]{Example}
\newtheorem{remark}[theorem]{Remark}

%\usepackage{lmodern}
% \usepackage[T1]{fontenc}

%\addtokomafont{section}{\mdseries\rmfamily\scshape}
% \addtokomafont{subsection}{\rmfamily\centering\scshape}


\title{INTRODUCTION TO DRINFIELD MODULES}
\begin{document}

\maketitle

\section{Introduction}
Our goal is to introduce Drinfeld modules and to explain their application to explicit class
field theory. First, however, to motivate their study, let us mention some of their applications.

\section{The Next Section}\label{sec1}
The function $\sin x$ can be defined as an infinite series
\begin{equation}\label{sineseries}
\sin x = x - \frac{x^3}{3!} + \frac{x^5}{5!} - \frac{x^7}{7!} + \cdots = \sum_{k \geq 0} \frac{x^{2k+1}}{(2k+1)!}.
\end{equation}
Here is another way to characterize it, using differential equations and initial conditions.

\begin{theorem}\label{diffthm}
The function $\sin x$ is the unique solution of the differential equation
\begin{equation}\label{sine-eqn}
\frac{d^2y}{dx^2} + y = 0
\end{equation}
satisfying the initial conditions $y(0) = 0$ and $y'(0) = 1$.
\end{theorem}

Notice in the code for this file that the number for the theorem, , is hard-coded, and that 
if you need to manually enter parentheses if you want the equation number to appear in text as (\ref{sine-eqn}).


\section{The Section After That}\label{sec2}

There is nothing here.

\end{document}



